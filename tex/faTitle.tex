% !TeX root=../main.tex
% در این فایل، عنوان پایان‌نامه، مشخصات خود، متن تقدیمی‌، ستایش، سپاس‌گزاری و چکیده پایان‌نامه را به فارسی، وارد کنید.
% توجه داشته باشید که جدول حاوی مشخصات پروژه/پایان‌نامه/رساله و همچنین، مشخصات داخل آن، به طور خودکار، درج می‌شود.
%%%%%%%%%%%%%%%%%%%%%%%%%%%%%%%%%%%%
% دانشگاه خود را وارد کنید
\university{دانشگاه تهران}
% پردیس دانشگاهی خود را اگر نیاز است وارد کنید (مثال: فنی، علوم پایه، علوم انسانی و ...)
\college{پردیس دانشکده‌های فنی}
% دانشکده، آموزشکده و یا پژوهشکده  خود را وارد کنید
\faculty{دانشکده‌ی برق و کامپیوتر}
% گروه آموزشی خود را وارد کنید (در صورت نیاز)
\department{گروه نرم‌افزار}
% رشته تحصیلی خود را وارد کنید
\subject{مهندسی کامپیوتر}
% گرایش خود را وارد کنید
\field{نرم‌افزار}
% عنوان پایان‌نامه را وارد کنید
\title{
تولید بافت برای مدل‌های سه‌بعدی اشیاء حاوی الگو‌های تکرارشونده
}
% نام استاد(ان) راهنما را وارد کنید
\firstsupervisor{دکتر هادی مرادی}
\firstsupervisorrank{دانشیار}
%\secondsupervisor{دکتر راهنمای دوم}
%\secondsupervisorrank{استادیار}
% نام استاد(دان) مشاور را وارد کنید. چنانچه استاد مشاور ندارید، دستورات پایین را غیرفعال کنید.
%\firstadvisor{دکتر مشاور اول}
%\firstadvisorrank{استادیار}
%\secondadvisor{دکتر مشاور دوم}
% نام داوران داخلی و خارجی خود را وارد نمایید.
\internaljudge{دکتر رشاد حسینی}
\internaljudgerank{استادیار}
%\externaljudge{دکتر داور خارجی}
%\externaljudgerank{دانشیار}
%\externaljudgeuniversity{دانشگاه داور خارجی}
% نام نماینده کمیته تحصیلات تکمیلی در دانشکده \ گروه
%\graduatedeputy{دکتر نماینده}
%\graduatedeputyrank{دانشیار}
% نام دانشجو را وارد کنید
\name{راستین}
% نام خانوادگی دانشجو را وارد کنید
\surname{سورکی}
% شماره دانشجویی دانشجو را وارد کنید
\studentID{810195410}
% تاریخ پایان‌نامه را وارد کنید
\thesisdate{بهمن ۱۴۰۰}
% به صورت پیش‌فرض برای پایان‌نامه‌های کارشناسی تا دکترا به ترتیب از عبارات «پروژه»، «پایان‌نامه» و «رساله» استفاده می‌شود؛ اگر  نمی‌پسندید هر عنوانی را که مایلید در دستور زیر قرار داده و آنرا از حالت توضیح خارج کنید.
%\projectLabel{پایان‌نامه}

% به صورت پیش‌فرض برای عناوین مقاطع تحصیلی کارشناسی تا دکترا به ترتیب از عبارت «کارشناسی»، «کارشناسی ارشد» و «دکتری» استفاده می‌شود؛ اگر نمی‌پسندید هر عنوانی را که مایلید در دستور زیر قرار داده و آنرا از حالت توضیح خارج کنید.
%\degree{}
%%%%%%%%%%%%%%%%%%%%%%%%%%%%%%%%%%%%%%%%%%%%%%%%%%%%
%% پایان‌نامه خود را تقدیم کنید! %%
%\dedication
%{
%{\Large تقدیم به:}\\
%\begin{flushleft}{
%	\huge
%	همسر و برادران \\
%	\vspace{7mm}
%	و\\
%	\vspace{7mm}
%	پدر و مادرم
%}
%\end{flushleft}
%}
%% متن قدردانی %%
%% ترجیحا با توجه به ذوق و سلیقه خود متن قدردانی را تغییر دهید.
\acknowledgement{
سپاس خداوندگار حکیم را که با لطف بی‌کران خود، آدمی را به زیور عقل آراست.

در ابتدا وظیفه‌ خود می‌دانم از زحمات بی‌دریغ استاد راهنمای خود، جناب آقای دکتر مرادی، صمیمانه تشکر و  قدردانی کنم که با اینکه بنده چندین بار در طول پژوهش از رسیدن به نتایج ناامید شده بودم، باز هم با صبوری بی‌مانند، راهنما و مشوق من بودند. یقینا بدون راهنمایی‌های ایشان این پژوهش به اتمام نمی‌رسید.

از جناب آقای دکتر حسینی و مهندس نصیری که  زحمت مشاوره‌ و کمک در این پژوهش را تقبل فرمودند نیز بسیار سپاسگزارم. از آقای مهندس حسینی که همیشه آماده‌ی راهنمایی و پاسخ دادن به سوالات من در ارتباط با پروژه‌ی ایشان بوده‌اند نیز کمال تشکر را دارم.

با سپاس فراوان خدمت دوستان گران‌مایه‌ام که در زمان ناامیدی و مشکلات، همیشه کنارم بودند و من را یاری دادند.

و در پایان، بسیار شاکرم از مهربانی‌ها، تشویق‌ها و حضور پدر و مادر عزیزم که بهترین پشتیبان من بودند.
}
%%%%%%%%%%%%%%%%%%%%%%%%%%%%%%%%%%%%
%چکیده پایان‌نامه را وارد کنید
\fa-abstract{
دو مسئله‌ی بازسازی سه‌بعدی و \gls{Texture Synthesis} از قدیم جزو مباحث پراهمیت در تحقیقات بینایی ماشین بوده‌اند. یکی از مباحث زیرمجموعه‌ی سه‌بعدی سازی، \gls{Single-View 3D Reconstruction} است که با پیشرفت‌های سخت‌افزاری و نرم‌افزاری جزو مباحثی است که مورد توجه زیادی قرار گرفته است. این مسئله به دو چالش اصلی تقسیم می‌شود: تولید \gls{Geometry}‌ی مدل سه‌بعدی و تولید بافت. امروزه استفاده از روش‌های مختلف تولید بافت، در مسئله‌ی بازسازی سه‌بعدی با استفاده از یک تصویر، به یکی از مباحث جذاب برای پژوهشگران در این زمینه تبدیل شده است. هدف این مسئله تولید بافت کامل مدل سه‌بعدی با استفاده از طرح رویی شیء است که از تصویر ورودی استخراج می‌شود. در این پژوهش ما تلاش می‌کنیم روشی برای تولید بافت مدل سه‌بعدی برای اشیاء حاوی الگو‌های تکرار‌شونده بر روی سطح خود، از یک تصویر، ارائه ‌دهیم. در این روش تلاش شده با استفاده از ویرایش تصاویر در \gls{Gradient Domain} و \gls{Poisson Blend}، اختلاف رنگ و شدت نور در نواحی مرزی قطاع‌های طرح تکرارشونده، به عنوان یک مسئله‌ی بهینه‌سازی خطی حل شود. در نتایج به‌دست آمده مشخص شد برای اشیاء حاوی الگو‌های تکرارشونده، می‌توان با استفاده از این روش، بافت‌هایی تولید کرد که از لحاظ بصری هموار‌تر و با اعوجاج‌های کمتری همراه باشند. تولید بافت برای اجسامی که تعداد تکرار الگو‌های تکرار‌شونده‌ی سطح آن‌ها فرد است، برتری این روش نسبت به تکرار طرح رویی برای قسمت پشت مدل بازسازی‌شده‌ی سه‌بعدی را نشان داد. دیگر مزیت این روش نسبت به روش آینه‌کردن تصویر برای همسان‌سازی شدت نور، توانایی حفظ قسمت‌های متنی روی طرح بوده است. به طور کلی استفاده از روش ارائه شده، می‌تواند به تولید مدل‌های سه‌بعدی بازسازی شده از اشیاء با ویژگی‌های بصری مناسب‌تری نسبت به دیگر روش‌ها بیانجامد. 
}

% کلمات کلیدی پایان‌نامه را وارد کنید
\keywords{
بینایی ماشین - بازسازی سه‌بعدی - تولید بافت - ویرایش تصاویر در فضای گرادیان
}
% انتهای وارد کردن فیلد‌ها
%%%%%%%%%%%%%%%%%%%%%%%%%%%%%%%%%%%%%%%%%%%%%%%%%%%%%%
